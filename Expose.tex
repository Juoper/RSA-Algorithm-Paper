\documentclass[14pt,a4paper]{scrartcl}
\usepackage[utf8]{inputenc}
\usepackage{amsmath, amssymb, amsfonts}
\usepackage{graphicx}
\usepackage[top=3cm, bottom=2.5cm, left=3cm, right=4cm]{geometry}\usepackage[english,german]{babel} 
\usepackage{csquotes}

\usepackage[style=ieee]{biblatex}
 
 
\parindent 0px
\linespread{1.25} %gleich wie 1.5 in word
\pagestyle{headings}

\addbibresource{ref.bib}

\begin{document} \selectlanguage{german}
\begin{titlepage}
 
  \null\vfill
  \begin{center}
    GYMNASIUM OTTOBRUNN
    \vskip 2em
    Oberstufenjahrgang 2022/2023
    \vskip 2em
    Seminar: Die Welt der Mathematik - die Mathematik in der Welt
    \vskip 1em
    Leitfach: Mathematik
    
    \vskip 5em
    Exposé zur Seminararbeit
    \vskip 2em
    
    {
      \usekomafont{title} \LARGE
        Der RSA - Algorithmus und seine Anwendung in der IT-Sicherheit
      \par
    }
     
    \vskip 1.5em
    {\usekomafont{date}{\today \par}}%
    \vskip 0pt plus 4fill
    
  \end{center}
  Verfasser: Julian Thanner
  \vskip 0.5em
  Seminarleiterin: OStRin Birgit Gregor
  \vskip 2em
  Bewertung: .............. Punkte
  \vskip 1em
  Unterschrift der Seminarleiterin: \underline{\hspace{5cm}}
 \end{titlepage}
	
\thispagestyle{empty}
\tableofcontents
\thispagestyle{empty}


\pagebreak
\section{Einleitung}
\subsection{Heutige Sicherheit im Internet}
Eigentlich ist es doch sehr erstaunlich, dass wir unser Geld einfach so über unser Handy verwalten können und keine Sorge haben müssen, dass jemand fremdes unerlaubt Dinge mit unserem Geld anstellt. Dies haben wir unter anderem dem RSA-Algorithmus zu verdanken, welcher einen großen Teil zur sicheren und geheimen Kommunikation im Internet beiträgt.

\subsection{Nachteile der symmetrischen Verschlüsselung}

Bei symmetrischen Verschlüsselungsverfahren gibt es, anders als bei asymmetrischen Verfahren (auch: Public Key Verfahren), nur einen Schlüssel für die Ent- und Verschlüsselung. Das Problem hierbei ist allerdings, sie sind nur dann sicher, wenn es möglich ist den Schlüssel sicher der anderen Seite zukommen zu lassen. In der Praxis ist das aber nur mit einem persönlichen Treffen möglich. Wenn man aber eine E-Mail an jemanden schicken möchte, mit dem man sich aktuell nicht physisch treffen kann, ist ein Schlüsselaustausch so nicht möglich.



\subsection{Notwendigkeit der Schlüsselübertragung}
[Hier Beispiel Alice Bob Truhe zwei Schlösser



\section*{Zeitplan}

%beginn ende was 
	\begin{tabular}{ c | c | p{10cm} }
	Beginn & Ende & Aufgabe\\
	 \hline

 & 14.6 & Expose abgegeben \\
14.6 & 28.6 & Vorbereitungen machen, wie Begriffserklärungen schreiben \\
28.6 & 28.7 & Fertig schreiben der Seminar Arbeit, da sonst wenig Zeit \\
28.8 & 17.9 & Korrektur lesen und Fehler korrigieren\\
18.9 & 20.10 & Puffer/Alles druckbereit und keine Fehler mehr\\
21.10 & 27.10 & Alles fertig und gedruckt
	\end{tabular}
	
	

\section{Begriffserklärungen}
	\subsection{Modulo Operation} %Notwendigkeit?
	\subsection{Einwegverschlüsselungsverfahren}
	\subsection{Eulersche $\phi$ Funktion}
	\subsection{Satz von Euler}
	\subsection{Der euklidische Algorithmus}
	\subsection{Zwei Arten von Verschlüsselung}
	\subsection{Hashen} %Rename

\section{Geschichte}
	\subsection{Erfindung}
	Der RSA Algorithmus wurde im September 1977 von R.L. Rivest, A. Shamir und L. Adleman erfunden, damit sicher gestellt werden kann, dass E-Mails privat sind und signiert werden können. Dieser entstand auf Basis des, von Diffie und Hellman entwickelten, Public-Key Konzept. \cite{1055638}. Dieses Konzept hat dann Rivest, Shamir und Adleman motiviert eine Implementierung dafür zu entwickeln, weil Sie keine konkrete Implementierung vorgestellt haben. \cite[2]{rsaOriginalPaper} 
%fragen ob das so indirekt zitiert passt

	\subsection{Zeitliche Entwicklung}
	\subsection{Verlauf in der IT Sicherheit}

\section{Der RSA - Algorithmus}

	Bemerkung
	\begin{enumerate}
	\item Die Buchstaben e und d leiten sich von den englischen Wörtern \glqq encryption" (Verschlüsselung) und \glqq decryption" (Entschlüssekung) ab.
		\end{enumerate}
		
	\subsection{Schlüsselgenerierung}
	\subsection{Ver- und Entschlüsselung}
	\subsection{Ganze Texte Verschlüsseln}

\section{Mögliche Angriffspunkte}
	\subsection{Primfaktorzerlegung}
	\subsection{Monoalphabetisch}

\section{Anwendungsbereiche in der IT- Sicherheit}
	\subsection{Nachrichten in Zahlen umwandeln}
	Um Nachrichten mit dem RSA Algorithmus zu verschlüsseln und damit ein Computerprogramm diese verarbeiten kann, müssen diese erst in Zahlen umgewandelt werden. Dafür existiert der ~ASCII~ Standard, mit dem Computer Buchstaben in Zahlen umwandeln können. Jede Nachricht \textit{m} muss kleiner sein als \textit{n}. In der Realität spielt das allerdings keine Rolle, da normalerweise Werte für \textit{n} mit mehr als 100 Stellen verwendet werden und der normale ASCII Standard 7-Bit Werte und der erweiterte ASCII Standard 8-Bit Werte verwendet.\footnote{\label{foot:1}7-Bit hat Zahlen bis 128, 8-Bit bis 256}
	
	\subsection{Digitale Signatur}
	
	Eine digitale Nachricht besteht aus Bits, die ein Computer in Daten, wie z.B. Text übersetzt. Diese Bits bestehen aus \glqq 0" und \glqq 1". \\ 
	Um nun eine Nachricht digital zu signieren, müssen die folgenden Schritte abgearbeitet werden, um eine digitale Signatur, die auch wieder aus Bits besteht, zu erhalten.\\
	Mithilfe des Hash-Algorithmus wird nun ein Hash-Wert mithilfe einer  der zu signierenden Nachricht berechnet. Dieser Wert wird dann mit dem privaten Schlüssel, dem Encryption-Key, der normalerweise der öffentliche Schlüssel ist, verschlüsselt. %Schöner formulieren
	Der Decryption-Key wird dann, anders als wenn man eine geheime Nachricht versenden will, veröffentlicht.\\
	Die originale Nachricht wird dann zusammen mit der digitalen Signatur verschickt und der Empfänger entschlüsselt diese dann mithilfe des öffentlichen Schlüssels wie folgt:\\
	Als erstes generiert der Empfänger mithilfe des Hash-Algorithmus einen Hash-Wert von der Nachricht, die auf ihre Echtheit überprüft werden soll. Dann wird mit dem öffentlichen Schlüssel des Senders der, an die Nachricht angehängte, verschlüsselte Hash-Wert des Senders entschlüsselt. \\
Wenn der generierte und der entschlüsselte Hash-Wert übereinstimmen, kann sicher festgestellt werden, dass die Nachricht nicht verändert wurde. \\
Wenn z.B. jetzt eine dritte Person die Nachricht abändern würde, würde ein anderer Hash-Wert beim Empfänger für diese Nachricht herauskommen, da der Hash-Algorithmus schon bei der Änderung eines einzelnen Bits einen anderen Hash-Wert ausgibt. Wenn dieser dann mit dem verschlüsselten Hash-Wert verglichen wird, kann festgestellt werden, dass die Nachricht verändert wurde.
	
	
	

	 	
	\subsection{Hybride Verschlüsselung}

\pagebreak
\section{Anhang}

\listoffigures
\pagebreak

\nocite{*}
\printbibliography
\pagebreak

\begin{flushleft}
Ich erkläre hiermit, dass ich die Seminararbeit ohne fremde Hilfe angefertigt und nur die im Literaturverzeichnis verwendeten Quellen verwendet habe.
\end{flushleft}

\end{document}
